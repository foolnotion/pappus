\documentclass[fontsize=12pt,a4paper]{scrartcl} % document type and language

\usepackage[utf8]{inputenc}
\usepackage{microtype}
\usepackage{amsmath}
\usepackage{lmodern}
\usepackage[proportional,scaled=1.064]{erewhon}
\usepackage[erewhon,vvarbb,bigdelims]{newtxmath}
\usepackage[T1]{fontenc}
\usepackage[ruled,linesnumbered,vlined]{algorithm2e}
\usepackage{hyperref}
\usepackage{cleveref}

\usepackage[margin=2cm]{geometry}

\setlength{\parindent}{0em}
\setlength{\parskip}{0.5em}
\renewcommand{\baselinestretch}{1.05}

\title{Affine Arithmetic}

\author{Bogdan Burlacu}


\begin{document}

%\titlehead{University of Applied Sciences Upper Austria, Josef Ressel Centre for Symbolic Regression}

\maketitle

\section{Introduction}

In affine arithmetic, a quantity $x$ is represented as the following affine form:

\begin{align}
    x = x_0 + x_1 \epsilon_1 + ... + x_n \epsilon_n
\end{align}

where $\epsilon_1, ..., \epsilon_n$ are symbolic real variables whose values are unknown but assumed to lie in $[-1, 1]$. Note that the number $n$ changes during the calculation.

In the case of a multivariate function $f=(x_1,...,x_m)$ the following affine forms are initialized:

\begin{align}
    x_1 &= \frac{\overline{x}_1 + \underline{x}_1}{2} +  \frac{\overline{x}_1 - \underline{x}_1}{2} \epsilon_1\\
    \vdots\\
    x_m &= \frac{\overline{x}_m + \underline{x}_m}{2} +  \frac{\overline{x}_m - \underline{x}_m}{2} \epsilon_m\\
\end{align}

where $[\underline{x}_k, \overline{x}_k]$ is the domain of variable $x_k$.

An affine form can be converted to an interval using the formula:

\begin{align}
    I(x) &= [x_0 - \Delta, x_0 + \Delta] & \text{where } \Delta = \sum_{i=1}^n |x_i|
\end{align}

%Let $I(x)$ be the conversion of an affine form to a corresponding standard interval.

\subsection{Linear operations}

For two affine forms, $x = x_0 + \sum_{i=1}^n x_i \epsilon_i$ and $y = y_0 + \sum_{i=1}^n y_i \epsilon_i$ the following linear operations are defined:

\begin{align}
    x \pm y &= (x_0 \pm y_0) + \sum_{i=1}^n (x_i \pm y_i) \epsilon_i\\
    x \pm \alpha &= (x_0 \pm \alpha) + \sum_{i=1}^n x_i \epsilon_i\\
    \alpha x &= (\alpha x_0) + \sum_{i=1}^n (\alpha x_i)
\end{align}

A nonlinear function $f(x)$ of an affine form is generally not able to be represented directly as an affine form. We must therefore consider a linear approximation of $f$ and a representation of the approximation error by introducing a new noise symbol $\epsilon_{n+1}$.

Let $X = I(x)$ be the range of $x$. For a nonlinear function $f(x)$, a linear approximation in the form $ax + b$ will have a maximum approximation error $\delta$:

\begin{align}
    \delta = \max_{x \in X} | f(x) - (ax + b) |
\end{align}

The result of the nonlinear operation can then be represented as follows:

\begin{align}
    f(x) &= ax + b + \delta \epsilon_{n+1}\\
         &= a(x_0 + x_1 \epsilon_1 + ... + x_n \epsilon_n) + b + \delta \epsilon_{n+1}
\end{align}

Nonlinear binomial operations are calculated similarly.

\section{Minima and maxima of multivariate functions}

We consider a multivariate nonlinear function
\begin{align}
    y = f(x_1, ..., x_m)
\end{align}

The domain of this function is the $m$-dimensional region (the box):
\begin{align}
    X^{(0)} &= \left( X_1^{(0)}, ..., X_m^{(0)}  \right)\\
            &= \left( [\underline{X_1^{(0)}}, \overline{X_1^{(0)}}], ..., [\underline{X_m^{(0)}}, \overline{X_m^{(0)}}] \right)
\end{align}

One of the first methods to calculate the bounds of the codomain of $f$ is Fujii's method, in which the maxima and minima are calculated with guaranteed accuracy by means of recursively dividing $X$ into subregions and applying interval arithmetic (IA) to bound the range of $f$ in each region. The method discards the subregions that are guaranteed not to contain the point corresponding to the minimum (maximum) value. 

\subsection{Miyajima and Kashiwagi's method}

Without loss of generality, we consider finding maxima of a two-dimensional function $f(x_1, x_2)$ in the box $X^{(0)} = (X_1^{(0)}, X_2^{(0)}) = ([\underline{X_1^{(0)}}, \overline{X_1^{(0)}}], [\underline{X_2^{(0)}}, \overline{X_2^{(0)}}])$.

For an interval $J$, let the center and the width of $J$ be $c(J)$ and $w(J)$, respectively.

For a box $X$, let $F_A(X)$ be the range boundary of $f$ in $X$ obtained by applying \emph{AA} and let the upper bound of $I(F_A(X))$ be $\overline{F_A(X)}$.

\begin{algorithm}
    \caption{Algorithm for computing maxima of multivariate function (part 1)}\label{alg:minima-maxima-p1}
    \KwData{$f(\mathbf{x})$, $X$ (domain of $f$), stopping criteria $\epsilon_r, \epsilon_b$}
    \KwResult{Maxima (minima) of $f$}
    
    \tcp{Step 1}
    Initialize lists $\mathcal{S}$ and $\mathcal{T}$ for storing boxes and range boundaries:\\
    $\mathcal{S} \gets \emptyset$\;
    $\mathcal{T} \gets \emptyset$\;
    
    \tcp{Step 2: divide $X^{(0)}$ into subregions $X^{(1)}$ and $X^{(2)}$}
    \uIf{$w(X_1^{(0)}) < w(X_2^{(0)})$}{
        $\begin{aligned}
            X^{(1)} &=([\underline{X_1^{(0)}}, \overline{X_1^{(0)}}], [\underline{X_2^{(0)}}, c(X_2^{(0)})])\\
            X^{(2)} &=([\underline{X_1^{(0)}}, \overline{X_1^{(0)}}], [c(X_2^{(0)}), \overline{X_2^{(0)}}])
        \end{aligned}$
    }\Else{
        $\begin{aligned}
            X^{(1)} &=([\underline{X_1^{(0)}}, c(X_1^{(0)})], [\underline{X_2^{(0)}}, \overline{X_2^{(0)}}])\\
            X^{(2)} &=([c(X_1^{(0)}), \overline{X_1^{(0)}}], [\underline{X_2^{(0)}}, \overline{X_2^{(0)}}])
        \end{aligned}$
    }

    \tcp{Step 3}
    Calculate $F_A(X^{(1)})$ and $F_A(X^{(2)})$, then calculate $\underline{f_{\max}^{(1)}}$ and $\underline{f_{\max}^{(2)}}$ (use \cref{alg:algorithm-1}). The lower bound of the maxima is then given as $\underline{f_{\max}} = \max(\underline{f^{(1)}_{\max}}, \underline{f^{(2)}_{\max})}$.
    
    \tcp{Step 4}
    \uIf{$\overline{F_A(X^{(1)})} < \underline{f_{\max}}$}{
        Insert $X^{(2)}$ and $F_A(X^{(2)})$ into $\mathcal{S}$ and discard $X^{(1)}$.
    }
    \uElseIf{$\overline{F_A(X^{(2)})} < \underline{f_{\max}}$}{
        Insert $X^{(1)}$ and $F_A(X^{(1)})$ into $\mathcal{S}$ and discard $X^{(2)}$.
    }
    \Else{
        Insert $X^{(1)}$, $F_A(X^{(1)})$, $X^{(2)}$, $F_A(X^{(2)})$ into $\mathcal{S}$.
    }
\end{algorithm}

\begin{algorithm}
    \caption{Algorithm for computing maxima of multivariate function (part 2)}\label{alg:minima-maxima-p2}
    \KwData{$f(\mathbf{x})$, $X$ (domain of $f$), stopping criteria $\epsilon_r, \epsilon_b$}
    \KwResult{Maxima (minima) of $f$}

    \tcp{Step 5}
    \While{$\mathcal{S} \neq \emptyset$}{
        %\tcp{5.1}
        Find the box $X^{(i)} \in \mathcal{S}$ for which $F_A(X^{(i)})$ is largest.\\
        \hspace{10em}$\begin{aligned}
            X^{(i)} = \arg\max_i \left( F_A(X^{(i)}) \right)
        \end{aligned}$\\
        %$X_{current} \gets X^{(i)}$\\
        Remove $X^{(i)}$ from $\mathcal{S}$.\\
        Select $X^{(i)}$ and $F_A(X^{(i)})$ as the box and range to be processed.\\
        %\tcp{5.2}
        Calculate $\underline{f_{\max}^{(i)}}$ (the candidates of $\underline{f_{\max}}$) by utilizing $X^{(i)}$ and $F_A(X^{(i)})$ and by applying \cref{alg:algorithm-1}. 
        Update $\underline{f_{\max}}=\max\{{\underline{f_{\max}^{(i)}}}\}$.\\
        %\tcp{5.3}
        Discard any box $X$ and range boundary $F_A(X)$ from $\mathcal{S}$ and $\mathcal{T}$ for which $\overline{F_A(X)} < \underline{f_{\max}}$.\\
        %\tcp{5.4}
    Narrow $X^{(i)}$ down by utilizing $X^{(i)}, F_A(X^{(i)})$ and $\underline{f_{\max}}$ using \cref{alg:algorithm-2}.\\
    %\tcp{5.5}
        Divide $X^{(i)}$ into $X^{(j)}$ and $X^{(k)}$.\\
        \uIf{$w(X_1^{(i)}) < w(X_2^{(i)})$}{
            \hspace{8em}$\begin{aligned}
                X^{(j)} &=([\underline{X_1^{(i)}}, \overline{X_1^{(i)}}], [\underline{X_2^{(i)}}, c(X_2^{(i)})])\\
                X^{(k)} &=([\underline{X_1^{(i)}}, \overline{X_1^{(i)}}], [c(X_2^{(i)}), \overline{X_2^{(i)}}])
            \end{aligned}$
        }\Else{
            \hspace{8em}$\begin{aligned}
                X^{(j)} &=([\underline{X_1^{(i)}}, c(X_1^{(i)})], [\underline{X_2^{(i)}}, \overline{X_2^{(i)}}])\\
                X^{(k)} &=([c(X_1^{(i)}), \overline{X_1^{(i)}}], [\underline{X_2^{(i)}}, \overline{X_2^{(i)}}])
            \end{aligned}$
        }
        %\tcp{5.6}
        Calculate $F_A(X^{(j)})$ and $F_A(X^{(k)})$.\\
        \uIf{$\max_{1 \leq h \leq m} w(X_h^{(j)}) < \epsilon_r$ \textbf{and} $w(I(F_A(X^{(j)}))) < \epsilon_b$}{
            Insert $X^{(j)}$ and $F_A(X^{(j)})$ into $\mathcal{T}$.
        }\Else{
            Insert $X^{(j)}$ and $F_A(X^{(j)})$ into $\mathcal{S}$.
        }
        \uIf{$\max_{1 \leq h \leq m} w(X_h^{(k)}) < \epsilon_r$ \textbf{and} $w(I(F_A(X^{(k)}))) < \epsilon_b$}{
            Insert $X^{(k)}$ and $F_A(X^{(k)})$ into $\mathcal{T}$.
        }\Else{
            Insert $X^{(k)}$ and $F_A(X^{(k)})$ into $\mathcal{S}$.
        }
    }
    \tcp{Step 6}
    Group together boxes in $\mathcal{T}$ that share a common point. Let $Y^{(1)},...,Y^{(l)}$ be one such group. Then, the maxima is given by $\cup_{h=1}^l I(F_A(Y^{(h)}))$, with corresponding point $\cup_{h=1}^l Y^{(h)}$. Repeat for all groups.
\end{algorithm}
    

\begin{algorithm}
    \caption{Algorithm 1}\label{alg:algorithm-1}
    \tcp{Compared to Fujii's method, this algorithm is able to calculate candidates bounding $f_{\max}$ more closely, therefore this allows to discard more subregions (boxes) in the initial stage.}

    Suppose $F_A(X)$ is calculated as follows:
    \begin{align}
        F_A(X) = a_0 + a_1 \epsilon_1 + ... + a_m + a_{m+1} + ... + a_n \epsilon_n\label{eq:fa}
    \end{align}
    Let the point (vector) $y=(y_1, ..., y_m)$ be as follows:
    \begin{align}
        y_i = 
        \begin{cases}
            \overline{X_i} & 0 < a_i\\
            \underline{X_i} & a_i < 0\\
            c(X_i) & \text{otherwise}.
        \end{cases}
        & (i=1,...,m)\\
    \end{align}
    Then, the candidate for $\underline{f_{\max}}$ is calculated as $f(y)$.
\end{algorithm}

\begin{algorithm}
    \caption{Algorithm 2}\label{alg:algorithm-2}
    Calculate $F_A(X)$ using \Cref{eq:fa}.\\
    Calculate
    \begin{align}
        \alpha = \sum_{i=m+1}^n |a_i|
    \end{align}\\
    \ForAll{$i=1,...,m$}{
        \uIf{$a_i \neq 0$}{
            Apply IA (interval arithmetic) as follows:
            \begin{align}
                \varepsilon_i^* = \frac{1}{a_i}\left(f_{\max} - a_0 - \alpha - \sum_{j=1,j \neq i}^m(a_j \times [-1, 1]) \right)
            \end{align}
        }\Else{
            Let $\varepsilon_i^* = [-1, 1]$.
        }
        Narrow $X_i$ down as follows:
        \uIf{$\varepsilon_i^* \in [-1, 1]$}{
            \hspace{4em}$\begin{aligned}
                X_i = \begin{cases}
                [\underline{X_i} + r(X_i)(\underline{\varepsilon_i^*}+1), \overline{X_i}] & 0 < a_i\\
                    [\underline{X_i}, \overline{X_i} - r(X_i)(1-\overline{\varepsilon_i^*})] & a_i < 0\\
                \end{cases} & \text{ where } r(X_i) = \frac{\overline{X_i} - \underline{X_i}}{2}
            \end{aligned}$
        }\uElseIf{$\underline{\varepsilon_i^*} \leq -1$ \textbf{and} $\overline{\varepsilon_i^*} \in [-1, 1)$ \textbf{and} $a_i < 0$}{
            \hspace{4em}$\begin{aligned}
                X_i = [\underline{X_i}, \overline{X_i} - r(X_i)(1-\overline{\varepsilon_i^*})]
            \end{aligned}$
        }\uElseIf{$\underline{\varepsilon_i^*} \in (-1, 1]$ \textbf{and} $1 \leq \overline{\varepsilon_i^*}$ \textbf{and} $0 < a_i$}{
            \hspace{4em}$\begin{aligned}
                X_i = [\underline{X_i} + r(X_i)(\underline{\varepsilon_i^*} + 1), \overline{X_i}]
            \end{aligned}$
        }\Else{
            We are not able to narrow $X_i$ down.
        }
    }
\end{algorithm}

\end{document}
